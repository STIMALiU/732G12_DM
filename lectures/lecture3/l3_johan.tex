\documentclass[10pt,english]{beamer}
%\documentclass[english,handout]{beamer} % For handouts
\input{../metropolis_preamble.tex}
\input{../macros.tex}
%\usepackage{extendedalt}
%\usepackage{animate} % Animations
%\usepackage{../lindsten}
%\usepackage{movie15}

\title{732G12 Data Mining}
\subtitle{Föreläsning 3}
\date{}
\author{Johan Alenlöv \\ IDA, Linköping University, Sweden}
\titlegraphic{\hfill\includegraphics[height=1.2cm]{../LiU_primary_black.pdf}}
%\institute{Joint work with\dots}


%% MY DEF %%
\newcommand{\itm}[1]{\mathrm{Item}_{#1}}
\newcommand{\pausa}{\pause}
%\renewcommand{\pausa}{}


\newenvironment{nscenter}
 {\parskip=0pt\par\nopagebreak\centering}
 {\par\noindent\ignorespacesafterend}

\begin{document}

\maketitle

\begin{frame}{Dagens föreläsning}
    
    \begin{itemize}
        \item Klassificering
        \item Trädmodeller
        \item Metoder för besultsträd
        \item Regularisering
    \end{itemize}

\end{frame}

\begin{frame}{Klassificering}
    \begin{greenbox}
        Målet är att dela in objekt i ett antal förutbestämda klasser.
    \end{greenbox}
\end{frame}


\end{document}