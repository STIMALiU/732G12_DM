\documentclass[10pt,english]{beamer}
%\documentclass[english,handout]{beamer} % For handouts
\input{../metropolis_preamble.tex}
\input{../macros.tex}
%\usepackage{extendedalt}
%\usepackage{animate} % Animations
%\usepackage{../lindsten}
%\usepackage{movie15}

\title{732G12 Data Mining}
\subtitle{Föreläsning 2}
\date{}
\author{Johan Alenlöv \\ IDA, Linköping University, Sweden}
\titlegraphic{\hfill\includegraphics[height=1.2cm]{../LiU_primary_black.pdf}}
%\institute{Joint work with\dots}


%% MY DEF %%
\newcommand{\itm}[1]{\mathrm{Item}_{#1}}
\newcommand{\pausa}{\pause}
%\renewcommand{\pausa}{}


\newenvironment{nscenter}
 {\parskip=0pt\par\nopagebreak\centering}
 {\par\noindent\ignorespacesafterend}

\begin{document}

\maketitle

\begin{frame}{Dagens föreläsning}
    
    \begin{itemize}
        \item Modellval
        \item Generaliserade linjära modeller
        \item Modellval för linjär regression
    \end{itemize}

\end{frame}

\begin{frame}{Modellval}
    
    \begin{itemize}
        \item Vi söker en modell som \textbf{generaliserar} väl.
        \begin{itemize}
            \item Med generalisering menas att den ska fungera bra på ny data.
        \end{itemize}
        \item En komplex modell har lättare att överanpassa.
        \item Vad är ''komplexitet''?
        \begin{itemize}
            \item Linjär modell: Antal variabler, interaktioner, transformationer etc.
            \item Neurala nätverk: Bredd och djup av modellen.
            \item Trädmodeller: Djupet.
        \end{itemize}
    \end{itemize}

\end{frame}

\begin{frame}{Regularisering}

    Regularisering är ett sätt att motverka överanpassning.

    \begin{itemize}
        \item Idé att hindra modellen att bli för komplex.
        \item Ger förhoppningsvis bättre generaliseringsfel.
        \item \textbf{Mycket viktigt tema inom maskininlärning}
        \item Görs på olika sätt för olika metoder.
    \end{itemize}

    Idé är att
    \begin{equation*}
        \text{Komplex modell} + \text{regularisering} = \text{en bra modell}
    \end{equation*}

    Kommer prata mer om detta senare.

\end{frame}

\begin{frame}{Regression och Klassificering}
    
    Två klassiska problem är regression och klassificering.

    \begin{itemize}
        \item Regression: Prediktera en variabel $y$, oftast $y$ kontinuerlig. Bruset $\varepsilon$ är:
        \begin{itemize}
            \item Vanligast är normalfördelat.
            \item Alternativt, t-fördelning, Gamma, Log-normal,\dots
            \item Kan också vara disktet, Poisson, Negativ binomial,\dots
        \end{itemize}
        \item Fördelningen ger felfunktionen.
        \item Klassificering: $y$ är kategorisk med 2 eller flera utfall:
        \begin{itemize}
            \item Binär: logistisk/probit regression
            \item Fler klasser: Multinomial logistisk/probit regression
            \item Kommer diskutera fler metoder senare
        \end{itemize}
        \item Hur skapa felfunktion?
    \end{itemize}

\end{frame}

\begin{frame}{Förväxlingsmatris}
\begin{center}
    \includegraphics[width=0.7\textwidth]{figs/Screenshot at 2020-08-18 12-24-31.png}
\end{center}

\begin{itemize}
    \item Precision:
    \begin{equation*}
        \operatorname{P} = \frac{f_{11} + f_{00}}{f_{11} + f_{10} + f_{01} + f_{00}}.
    \end{equation*}
    \item Felkvot (error rate):
    \begin{equation*}
        \operatorname{E} = \frac{f_{10} + f_{01}}{f_{11} + f_{10} + f_{01} + f_{00}}.
    \end{equation*}
\end{itemize}
    
\end{frame}

\begin{frame}{Förväxlingsmatris}
    \begin{center}
        \includegraphics[width=0.7\textwidth]{figs/Screenshot at 2020-08-18 12-24-31.png}
    \end{center}
    
    \begin{itemize}
        \item Sensitivitet (recall, hit rate, true positive rate):
        \begin{equation*}
            \operatorname{TPR} = \frac{f_{11} }{f_{11} + f_{10}}.
        \end{equation*}
        \item Specifitet (selectivity, true negative rate):
        \begin{equation*}
            \operatorname{TNR} = \frac{f_{00}}{f_{00} + f_{01}}.
        \end{equation*}
    \end{itemize}
        
\end{frame}

\begin{frame}{F-score}
    
    \begin{itemize}
        \item F-score är ett mått av träffsäkerhet baserat på precision och sensitivitet.
        \begin{equation*}
            F_{\beta} = (1 + \beta^2) \frac{\operatorname{P} \cdot \operatorname{TPR}}{\beta^2 \cdot \operatorname{P} + \operatorname{TPR}}
        \end{equation*}
        \item Vanligt med $\beta=1$ vilket ger (harmoniska medelvärdet)
        \begin{equation*}
            F_1 = 2 \frac{\operatorname{P} \cdot \operatorname{TPR}}{\operatorname{P} + \operatorname{TPR}}
        \end{equation*}
        \item F-score är mellan 0 och 1, högre är bättre.
        \item $\beta$ säger hur du värderar precision och sensivitet.
        \item Beräknas klassvis.
    \end{itemize}

\end{frame}

\begin{frame}{Generaliserade linjära modeller}

    Antag att data $\mathbf{y} = y_1, y_2, \ldots, y_n$ är oberoende observationer från sannolikhetsfördelning från exponentialfamiljen.

    Vi har en linjär prediktor $\mathbf{X} \beta$ samt en länkfunktion $g$ som kopplar ihop prediktorn med medelvärdet $\mu$ genom
    \begin{equation*}
        g(\mu) = \mathbf{X} \beta.
    \end{equation*}
    
    Tar "vanlig" linjär regression till andra typer av responsvariabler
    \begin{itemize}
        \item Kontinuerlig: Normal, t, gamma, log-normal
        \item Binär: Logistisk regression
        \item Nominell: Multinomiell logistisk regression
        \item Frekvensdata: Poisson regression
    \end{itemize}
\end{frame}

\begin{frame}{Generaliserade linjära modeller - Exempel}

    \myheading{Linjär regression}
    \begin{itemize}
        \item Likelihood: Normal
        \item L\"ankfunktion: identitetsfunktionen
        \item SKattas genom att minimera:
        \begin{equation*}
            \operatorname{RSS} = \sum_{i=1}^{n} \left( y_i - \beta_0 - \sum_{j=1}^{p} \beta_j x_{ij} \right)^2
        \end{equation*}
    \end{itemize}

\end{frame}

\begin{frame}{Generaliserade linjära modeller - Exempel}
    \myheading{Logistisk regression}
    \begin{itemize}
        \item Likelihood: Bernoulli ($y$ kan vara 0 eller 1, $\mathbb{P}(y = 1) = p$)
        \item L\"ankfunktion: Logit,
        \begin{equation*}
            g(\mu) = \log\left( \frac{\mu}{1-\mu} \right)
        \end{equation*}
        \item Skattas med att maximera MLE
    \end{itemize}
\end{frame}

% \begin{frame}{Generaliserade linjära modeller - Exempel}
%     \myheading{Multinominell logistisk regression}
%     \begin{itemize}
%         \item Likelihood: Kategorisk ($y$ kan ta $K$ olika värden, $\mathbb{P}(y = k) = p_k$)
%         \item L\"ankfunktion: Logit,
%         \begin{equation*}
%             g(\mu) = \log\left( \frac{\mu}{1-\mu} \right)
%         \end{equation*}
%         \item Skattas med att maximera MLE
%     \end{itemize}
% \end{frame}

\begin{frame}{Modellval för linjära regression}

    Utgå ifrån: $y$ kontinuerlig med normal likelihood.

    Vi har ett antal förklarande variabler $\mathbf{x} = (x_1, \ldots, x_p)$. Vill hitta parametrar/modell som ger minst generaliseringsfel.

    Två alternativ:
    \begin{itemize}
        \item Alternativ 1: Välj ut en delmängd av variablerna.
        \begin{itemize}
            \item Best subset, forward selection, backward selection
        \end{itemize}
        \item Alternativ 2: Behåll alla variabler men begränsa parametrarna.
        \begin{itemize}
            \item Regularisering, Ridge och Lasso.
        \end{itemize}  
    \end{itemize}
    
\end{frame}

\begin{frame}{Utvärderingsmått}
    Om vi har flera modeller, hur jämför vi dessa?

    Två alternativ:
    \begin{itemize}
        \item Indirekt skatta testfelet
        \begin{itemize}
            \item Utgå från träningsmängden
            \item Försök att minska den bias som uppstår när vi inte använder all data.
        \end{itemize}
        \item Direkt skatta testfelet
        \begin{itemize}
            \item Valideringsdata
            \item Korsvalidering
        \end{itemize}
    \end{itemize}
\end{frame}

\begin{frame}{Indirekt skatta testfelet}
    MSE på träningsdatan underskattar "riktiga" MSE värdet, kan inte användas för att välja modell.

    Idé: justera träningsfelet för att ta hänsyn till detta. 

    \begin{equation*}
        C_p = \frac{1}{n} (\operatorname{RSS} + 2 d \hat{\sigma}^2),
    \end{equation*}

    Litet $C_p$ är bäst.

    \begin{equation*}
        \text{adjusted R}^2 = 1 - \frac{\operatorname{RSS}/(n-d-1)}{\operatorname{TSS}/(n-1)},
    \end{equation*}
    där
    \begin{equation*}
        \operatorname{TSS} = \sum_{i=1}^{n}(y_i - \bar{y})^2.
    \end{equation*}

    Stort $\text{adjusted R}^2$ är bäst.
\end{frame}

\begin{frame}{Indirekt skatta testfelet}
    
    Tre andra mått AIC, BIC och HQIC är baserade på MLE skattning ev modeller.

    Låt $\log(\hat{L})$ vara log-likelihood för optimala parametervärden.

    \begin{align*}
        \operatorname{AIC} &= 2 d - 2 \log(\hat{L}) \\
        \operatorname{BIC} &= d \cdot \log(n) - 2 \log(\hat{L}) \\
        \operatorname{HQIC} &= d \cdot \log(\log(n)) - 2 \log(\hat{L}) 
    \end{align*}

    Lågt värde är bättre.

\end{frame}

\begin{frame}{Indirekt skatta testfelet}
    
    \myheading{Linjär Regression}

    \begin{align*}
        \operatorname{AIC} &= \frac{1}{n \hat{\sigma}^2} (\operatorname{RSS} + 2 d \hat{\sigma}^2) \\
        \operatorname{BIC} &= \frac{1}{n \hat{\sigma}^2} (\operatorname{RSS} + \log(n) d \hat{\sigma}^2) \\
        \operatorname{HQIC} &= \frac{1}{n \hat{\sigma}^2} (\operatorname{RSS} + \log(\log(n)) d \hat{\sigma}^2)
    \end{align*}

    För linjär regression är $C_p \propto \operatorname{AIC}$.

    \includegraphics[width = \textwidth]{figs/logplot.png}

\end{frame}

\begin{frame}{Best subset selection}
    \begin{nscenter}
        \includegraphics[width=\textwidth]{figs/best subset selection.png}
    \end{nscenter}
    Bild från kursboken "An Introduction to Statistical Learning with Applications in R".

\end{frame}

\begin{frame}{Forward selection}
    \begin{nscenter}
        \includegraphics[width=\textwidth]{figs/Forward stepwise selection.png}
    \end{nscenter}
    Bild från kursboken "An Introduction to Statistical Learning with Applications in R".

\end{frame}

\begin{frame}{Backward selection}
    \begin{nscenter}
        \includegraphics[width=\textwidth]{figs/Backward stepwise selection.png}
    \end{nscenter}
    Bild från kursboken "An Introduction to Statistical Learning with Applications in R".

\end{frame}

\begin{frame}{Krympning (Shrinkage)}
    
    Idé: begränsa hur stora parametrarna får vara.
    \begin{itemize}
        \item Straffa stora parametrar
        \item Ändra deras värdemängd
    \end{itemize}
    Vanligaste metoderna:
    \begin{itemize}
        \item Ridge: $l^2$-norm
        \item Lasso: $l^1$-norm
    \end{itemize}

    Kom ihåg att standardisera era förklarande variabler först innan krympning!

\end{frame}

\begin{frame}{Ridge Regression}

    I vanliga regression minimerar vi
    \begin{equation*}
        f(\beta) = \operatorname{RSS} = \sum_{i=1}^{n}\left(y_i - \beta_0 - \sum_{j=1}^{p}\beta_j x_{ij}\right)^2
    \end{equation*}
    I Ridge lägger vi till $l^2$-norm på $\beta$ vilket ger
    \begin{equation*}
        f_{\text{Ride}}(\beta) = \sum_{i=1}^{n}\left(y_i - \beta_0 - \sum_{j=1}^{p}\beta_j x_{ij}\right)^2 + \lambda \sum_{j=1}^{p} \beta_j^2, \quad \lambda \geq 0.
    \end{equation*}
    $\lambda$ är en \textbf{hyperparameter} som vi behöver sätta.

    Notera att $\beta_0$ inte påverkas.
    
\end{frame}

\begin{frame}{Ridge Regression}

    \includegraphics[width=\textwidth]{figs/standardized ridge regression coefficients.png}
    
\end{frame}

\begin{frame}{Lasso Regression}
    I vanliga regression minimerar vi
    \begin{equation*}
        f(\beta) = \operatorname{RSS} = \sum_{i=1}^{n}\left(y_i - \beta_0 - \sum_{j=1}^{p}\beta_j x_{ij}\right)^2
    \end{equation*}
    I Ridge lägger vi till $l^1$-norm på $\beta$ vilket ger
    \begin{equation*}
        f_{\text{Ride}}(\beta) = \sum_{i=1}^{n}\left(y_i - \beta_0 - \sum_{j=1}^{p}\beta_j x_{ij}\right)^2 + \lambda \sum_{j=1}^{p} |\beta_j|, \quad \lambda \geq 0.
    \end{equation*}
    $\lambda$ är en \textbf{hyperparameter} som vi behöver sätta.

    Notera att $\beta_0$ inte påverkas.
\end{frame}

\begin{frame}{Lasso Regression}
    
    \includegraphics[width=\textwidth]{figs/standardized lasso coefficients.png}

\end{frame}

\begin{frame}{Ridge vs Lasso}
    \begin{tabular}{lll}
        & Ridge & Lasso \\ \hline
        $\lambda \to 0$ & $\hat{\beta}_{\text{Ridge}} \to \hat{\beta}_{\text{OLS}}$ & $\hat{\beta}_{\text{Lasso}} \to \hat{\beta}_{\text{OLS}}$ \\
        $\lambda \to \infty$ &  $\hat{\beta}_{\text{Ridge}} \to 0$ &  $\hat{\beta}_{\text{Lasso}} \to 0$ \\
        Norm & $l^2$-norm & $l^1$-norm \\
        Område & Hypersfär & Polytop \\
        Parametrar & Krymper och ger många små & Sätter många till 0 \\
        Korrelerade variable & Krymper alla lika mycket & Sätter en till 0
    \end{tabular}
\end{frame}

\begin{frame}{Ridge vs Lasso}

    \includegraphics[width=\textwidth]{figs/Contours of the error .png}
    
\end{frame}

\begin{frame}{Ridge vs Lasso}
    
    \begin{itemize}
        \item I R, skattas enkelt med \texttt{glmnet()} eller \texttt{cv.glmnet()}.
        \item Vilken som är bäst beror på kontext.
        \item Ridge passar bra när $y$ beror på de flesta variablerna.
        \begin{itemize}
            \item Många variabler och ungefär samma effektstorlek.
        \end{itemize}
        \item Lasso passar bra när $y$ beror på bara några få variabler.
        \begin{itemize}
            \item Fåtal variabler med hög effektstorlek, resten nära 0.
        \end{itemize}
        \item Lasso kan vara lättare att tolka.
        \item Lättare att göra inference på parameterarna i Ridge.
        \item Vilken man ska välja får ofta avgöras empiriskt för specifika dataset.
        \item Finns många utökningar.
    \end{itemize}

\end{frame}

\begin{frame}{Adaptive-Lasso}
    \begin{itemize}
        \item Låt varje parameter $\beta_j$ ha sin egen vikt $w_j$
        \begin{equation*}
            f(\beta) = \sum_{i=1}^{n}\left(y_i - \beta_0 - \sum_{j=1}^{p}\beta_j x_{ij}\right)^2 + \lambda \sum_{j=1}^{p} w_j |\beta_j|, \quad \lambda \geq 0.
        \end{equation*}
        \item Välj vikter
        \begin{equation*}
            w_j = | \hat{\beta}_{j, \text{start}} |^{-\gamma}
        \end{equation*}
        \item Ofta väljs $\gamma = 1$.
        \item Behöver startvärde $\hat{\beta}_{j,\text{start}}$, kan använda OLS eller Ridge skattning.
        \item Stora startvärden ger små vikter vilket gör att parametern krymper mindre.
        \item Finns vissa teorestiska fördelar, men kräver extra skattning.
    \end{itemize}
\end{frame}

\begin{frame}{Elasticnet regression}

    \begin{itemize}
        \item Kombinerar Ridge och Lasso.
        \item Ny hyperparameter $\alpha$ för att mixa dessa,
        \begin{equation*}
            f(\beta) = \operatorname{RSS} + \lambda \left((1-\alpha) \sum_{j=1}^{p}\beta_j^2 + \alpha \sum_{j=1}^{p}|\beta_j|  \right), \qquad \lambda \geq 0,\, 0 \leq \alpha \leq 1.
        \end{equation*}
        \item $\lambda$ är likt tidigare hur mycket regularisering vi gör.
        \item $\alpha$ väljer hur mycket vikt vid Ridge respektive Lasso.
        \item Kan tvinga vissa $\beta$ till 0, men inte lika många som Lasso.
        \item Klarar av korrelerade/grupper av variabler bättre än Lasso.
        \item Nackdel är att vi har två hyperparametrar.
    \end{itemize}
    
\end{frame}


\end{document}