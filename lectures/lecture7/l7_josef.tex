\documentclass[10pt,english]{beamer}
%\documentclass[english,handout]{beamer} % For handouts
\input{../metropolis_preamble.tex}
\input{../macros.tex}
%\usepackage{extendedalt}
%\usepackage{animate} % Animations
%\usepackage{../lindsten}
%\usepackage{movie15}
\usepackage{tikz}
\usepackage{listofitems} % for \readlist to create arrays

\hypersetup{
  colorlinks=true, urlcolor=blue, linkcolor=red
}

\title{732G12 Data Mining}
\subtitle{Föreläsning 7}
\date{}
\author{Josef Wilzén \\ IDA, Linköping University, Sweden}
\titlegraphic{\hfill\includegraphics[height=1.2cm]{../LiU_primary_black.pdf}}
%\institute{Joint work with\dots}


%% MY DEF %%
\newcommand{\itm}[1]{\mathrm{Item}_{#1}}
\newcommand{\pausa}{\pause}
%\renewcommand{\pausa}{}
\tikzstyle{mynode}=[thick,draw=blue,fill=blue!20,circle,minimum size=22]


\newenvironment{nscenter}
 {\parskip=0pt\par\nopagebreak\centering}
 {\par\noindent\ignorespacesafterend}

\begin{document}

\maketitle

\begin{frame}{Dagens föreläsning}

    \begin{itemize}
        \item Neurala nätverk
        \item Feature learning
        \item Optimering av neurala nätverk
        \item Hyperparametrar
    \end{itemize}
    
\end{frame}


\begin{frame}{Neuralt nätverk}
    När vi pratar om neurala nätverk kan vi prata om lite vad som helst. Finns \textbf{väldigt många} olika sorters nätverk. Se \href{https://www.asimovinstitute.org/neural-network-zoo/}{här} för en sammaställning.

    Kan använda neurala nätverk för att lösa många olika problem:
    \begin{itemize}
        \item Övervakad inlärning
        \item Oövervakad inlärning
        \item Reinforcement learning
        \item Generativa modeller
        \item Representation learning
    \end{itemize}

\end{frame}


\begin{frame}{Neurala nätverk}
    
    För övervakad inlärning kan vi t.ex. använda
    \begin{itemize}
        \item Feed-forward network / Multiple layer perceptron (MLP)
        \item Radial basis network
        \item Faltade (Convolutional) nätverk (CNN): bilder, videor, tidserier.
        \item Recurrent neural networks, LSTM
        \item Transformer networks 
    \end{itemize}

\end{frame}


\begin{frame}{Neurala nätverk}
    Feed-forward nätverk: Inlager $\rightarrow$ Gömda lager $\rightarrow$ Utlager
    \includegraphics[scale=0.15]{NN_image1.png}

\end{frame}


\begin{frame}{Neurala Nätverk}

    För oövervakad inlärning:
    \begin{itemize}
        \item Dolda representationer: Autoencoders, Variational autoencoders
        \item Clustering: Self Organizing Map
    \end{itemize}

    Generativa modeller:
    \begin{itemize}
        \item Används för att lära sig komplexa fördelningar för att sen dra nya samples.
        \begin{itemize}
            \item Sampla nya bilder
            \item Skriva text
        \end{itemize}
        \item Generative adversarial network (GAN)
        \item Exempel: chatGPT, Gemini, Llama, stable diffusion
    \end{itemize}
    
\end{frame}

\begin{frame}{Feature learning}

    Vi går tillbaka till linjär regression,
    \begin{equation*}
        y = \mathbf{X} \beta + \varepsilon, \quad \mathbb{E}[\varepsilon] = 0, \quad \mathbb{V}[\varepsilon] = \sigma^2.
    \end{equation*}
    
    Vad kan vi göra om data inte följer denna linjära modell?

\end{frame}

\begin{frame}{Feature learning}
    
    Vanligt i linjär regression,
    \begin{itemize}
        \item Givet data $\mathbf{X} = (x_1, x_2, \ldots, x_p)$ och $ y = \mathbf{X} \beta$.
        \item Vi kan transformera variablerna i $\mathbf{X}$:
        \begin{itemize}
            \item Polynomregression, $\mathbf{X} = (x, x^2, x^3, \ldots, x^p)$
            \item Funktioner, $\log(x), \sqrt{x}, \exp(x), \ldots$
            \item Interaktioner, $x_1 x_2$
            \item Stegfunktioner
            \item Disktretisering
            \item Dummy-kodning
        \end{itemize}
        \item Kallas i maskininlärning för "feature engineering"
        \begin{itemize}
            \item Svårt att veta vilken transformation som vi ska göra för varje problem.
            \item Svårt med komplexa datastrukturer som text eller bild.
        \end{itemize}
    \end{itemize}

\end{frame}

\begin{frame}{Feature learning}
    
    \begin{itemize}
        \item Vi har data $\mathbb{X} = (x_1, x_2, \ldots, x_p)$.
        \item Transformationer är funktioner av data.
        \begin{itemize}
            \item Ex. $h(x) = \log(x), h(x_1,x_2) = \exp(x_1 \cdot x_2)$.
        \end{itemize}
        \item Anta en $x$-variabel, låt $h(x)$ vara en viktad summa av andra funktioner,
        \begin{equation*}
            z = h(x) = \sum_{i=1}^{M}w_i h_i(x),
        \end{equation*}
        där $h_i(x)$ är godtyckliga funktioner.
        \item<2-> Om vi har många $x$-variabler får vi,
        \begin{equation*}
            z =  h(x_1, x_2, \ldots, x_p) = \sum_{i=1}^{M} w_i h_i(x_1, x_2, \ldots, x_p).
        \end{equation*}
        \item<2-> Hur ska vi välja $h_i(x)$?
    \end{itemize}

\end{frame}

\begin{frame}{Feature learning}
    
    För en linjär transformation hitta matriserna $\mathbf{W}$ och $\mathbf{V}$,
    \begin{equation*}
        \underset{n \times m}{\mathbf{Z}} = \underset{n \times p}{\mathbf{X}} \cdots \underset{p \times m}{\mathbf{W}}, \qquad \underset{n \times g}{\mathbf{Z}} = \underset{n \times p}{\mathbf{X}} \cdots \underset{p \times m}{\mathbf{W}} \cdot \underset{m \times g}{\mathbf{V}}.
    \end{equation*}

    För neurala nätverk vill vi kunna modellera icke-linjära funktioner.

    Idé: Använd många "enkla" icke-linjära funktioner för att skapa en komplex icke-linjär funktion!

\end{frame}

\begin{frame}{Neurala nätverk}

    \begin{greenbox}

        Låt $\sigma(\cdot)$ vara en enkel icke-linjär funktion som opererar elementvis och låt $h_i(x_1, \ldots, x_p)$ vara en linjär funktion,
        \begin{equation*}
            h_i(x_1, x_2, \ldots, x_p) = \beta_{0i} + \beta_i^{\top} \mathbf{x}.
        \end{equation*}

        Låt nu
        \begin{equation*}
            z = \sigma(h_i(x_1, x_2, \ldots, x_p)) = \sigma(\beta_{0i} + \beta_i^{\top} \mathbf{x}),
        \end{equation*}
        nästla sedan många sådana funktioner för att bygga upp en godtyckligt komplex icke-linjär funktion.
    \end{greenbox}
    
\end{frame}

\begin{frame}{Definiera lager}
    
    För MLP brukar vi skriva
    \begin{equation*}
        \underset{k \times 1}{a}^{(p+1)} = \sigma \left( \underset{k \times n}{\mathbf{W}}^{(p)} \cdot \underset{n \times 1}{a}^{(p)} + \underset{k \times 1}{b}^{(p)} \right).
    \end{equation*}
    Här är:
    \begin{description}
        \item[$k$] Dimension av nya lagret
        \item[$n$] Dimension av föregående lager
        \item[$\mathbf{W}$] Viktmatris
        \item[$b$] Bias
        \item[$(p)$] Vilket lager
        \item[$\sigma()$] Vår funktion som opererar elementvis   
    \end{description}
    Historiskt har sigmoid eller hyperbolic tangent varit vanliga aktiveringsfunktioner. Numera är ReLu (eller varianter) den vanligaste,
    \begin{equation*}
        \operatorname{ReLu}(x) = \max(0,x).
    \end{equation*}
\end{frame}


\begin{frame}{Sista lagret - output layer}
    
    Hur sista lagret ser ut och vilken aktivetsfunktion som används beror på repsonsvariabeln.
    
    Regression:
    \begin{itemize}
        \item $y\in \mathbb{R}$ $\rightarrow$ använd identitetsfunktionen
        \item $y\in \mathbb{R}^+$ $\rightarrow$ använd $exp()$ eller softplus $log(1+exp(x))$
        \item Om $y$ är multivariat $\rightarrow$  låt utlagret ha flera noder
    \end{itemize}
    
    Klassificering
    \begin{itemize}
        \item $y$ binär $\rightarrow$ sigmoid $1 / (1+exp(-x))$ - flera sigmoid-noder kan användas om y är multivariat
        \item $y$ nominell $\rightarrow$ en nod per klass $\rightarrow$ Softmaxfunktionen:
          \begin{equation*}
            \sigma(\bold{z})_i = \frac{exp(z_i)}{\sum_{j=1}^{K}exp(z_j)}
          \end{equation*}
    \end{itemize}
    
    Notera: ett nätverk kan utföra regression och klassificering samtidigt om vi vill!
    
\end{frame}


\begin{frame}{Neurala nätverk}

    Vi kan se ett neuralt nätverk som att vi
    \begin{enumerate}
        \item Automatiskt lär oss transformationer av de förklarande variablerna.
        \item Gör linjär (logistisk,multinomiell) regression på transformationerna (sista lagret).
    \end{enumerate}
    OBS!
    \begin{itemize}
        \item Komplexa funktioner kräver mycket data att lära sig!
        \item Neurala nätverk kan lätt överanpassa träningsdata!
        \item Funkar när vi har stort antal förklarande variabler.
        \item Om vi låter gömda lager ha mindre dimension än förklarande variabler får vi "icke-linjär variabelreduktion".
        \item "Black box"
    \end{itemize}
    
\end{frame}

\begin{frame}{Universal approximation theorem}
    Vilka funktioner kan vi då lära oss med ett neuralt nätverk av detta slag?

    \begin{greenbox}
        \myheading{Universal approximation theorem} (informellt): En MLP med ett lager och en icke-linjär aktiveringsfunktion kan approximera godtycklig kontinuerlig eller diskret funktion med ett godtyckligt litet fel givet tillräckligt många gömda neuroner.
    \end{greenbox}
\end{frame}

\begin{frame}{Optimering av neurala nätverk}
    
    Gradient decent: Hitta minimum på en funktion genom att gå dit den lutar mest!

    Vill hitta
    \begin{equation*}
        a^* = \argmin_a L(a) = \sum_i L_i(f(x^{(i)},a), y^{(i)}).
    \end{equation*}

    där $a$ är nätverkets parametrar och $L(a)$ är kostandsfunktionen. Löser detta genom sekvensen
    \begin{equation*}
        a_{n+1} = a_n - \gamma \cdot \nabla L(a_n).
    \end{equation*}

    \begin{itemize}
        \item Vi behöver gradienter (partiella derivator) 
        \item Backpropagation: kedjeregel för derivator på neruala nätverk.
        \item Gradient decent: dyrt när vi har många observationer!
    \end{itemize}

\end{frame}

\begin{frame}{Optimering av neurala nätverk}
    
    Svårt problem med många fallgropar.

    \begin{itemize}
        \item Lokala minima
        \begin{itemize}
            \item Ställen som ser ut som ett minima (eller grannar har högre kostand) men inte är det bästa som finns.
            \item Kan ha hög kostand eller låg.
            \item Identifikationsproblem:
            \begin{itemize}
                \item Viktsymmetri $\rightarrow$ annan ordning på gömda noder ger samma modell/output
                \item Skalning mellan lager
            \end{itemize}
            \item Kan ha oräkneligt antal lokala minima.
        \end{itemize}
    \end{itemize}

\end{frame}

\begin{frame}{Optimering av neurala nätverk}
    
    Platåer och sadelpunkter
    \begin{itemize}
        \item Ställen där gradienten är noll (eller nära), fast vi inte är på ett loaklt min/max.
        \item Sadelpunkter:
        \begin{itemize}
            \item Lokalt minima i några riktningar.
            \item Lokalt maxima i andra riktningar.
        \end{itemize}
        \item Antalet sadelpunkter tenderar att öka med antalet dimensioner.
        \item Platåer är stora områden som är platta (gradient nära noll).
        \item Platåer och sadelpunkter gör optimeringen med gradient decent svårare.
    \end{itemize}

\end{frame}

\begin{frame}{Optimering av neurala nätverk}
    
    Exempel på sadelpunkt i två dimensioner:
    
    \includegraphics[scale=0.2]{Saddle_point.svg.png}

\end{frame}

\begin{frame}{Stochastic gradient decent (SGD)}

    Det är dyrt att beräkna $\nabla L(a_n)$ för alla datapunkter.

    Gör istället en väntesvärdesriktig skattning $\nabla \hat{L}(a_n)$ av gradienten genom att ta ett slumpmässigt urval från data (mini-batch).

    \begin{itemize}
        \item Större batch ger mindre varians i skattningen men blir dyrare att beräkna.
        \item Kräver fler iterationer och mindre learning rate.
        \item Kräver att vi har oberoende observationer.
        \item Funkar bra för neurala nätverk!
        \item En epok (epoch) är en genomgång av all träningsdata i SGD.
    \end{itemize}
    
\end{frame}

\begin{frame}{Hyperparametrar}
    
    I neruala nätverk finns massvis med hyperparametrar!
    \begin{itemize}
        \item Arkitektur:
        \begin{itemize}
            \item Antal gömda lager
            \item Antal neuroner i varje lager
            \item Aktiveringsfunktioner
            \item (Speciella typer av neuroner/lager)
        \end{itemize}
        \item Optimeringen:
        \begin{itemize}
            \item Mini-batch storlek
            \item Learning rate (fix eller föränderlig)
            \item Antal epoker
            \item (Vilken optimeringsalgoritm som används)
        \end{itemize}
    \end{itemize}

\end{frame}

\begin{frame}{Hyperparametrar}
    Hur ska vi bestämma deras värden?
    \begin{itemize}
        \item Svår fråga utan exakt svar.
        \item Mycket trial and error.
        \item Valideringsdata
        \item Använda föreslagna nätverksarkitekturer från litteraturen
        \begin{itemize}
          \item Finns en mängd förtränade modeller som kan anpassas för nya problem/data
        \end{itemize}
        \item För stora problem kan det ta lång tid att hitta bra hyperparametrar
    \end{itemize}
\end{frame}

\begin{frame}{Kommentarer - neurala nätverk }
    \begin{itemize}
        \item Mycket flexibelt ramverk för att modellera data
          \begin{itemize}
            \item Många olika kostandsfunktioner för olika situationer
            \item Många specialiserade arkitekturer
            \item Klarar av många olika sorts data
          \end{itemize}
        \item Har ofta bra prediktiv förmåga
        \item Finns risk för överanpassning
        \item Funkar ofta bra med många observationer och variabler
        \begin{itemize}
                  \item Tabelldata: på mindre dataset så funkar ofta andra enklare metoder lika bra eller bättre.
        \end{itemize}
        \item Låg tolkningsbarhet på de flesta nätverk: "black box modeling"
        \item Kan ta lång tid att träna och bestämma hyperparametrar
    \end{itemize}
\end{frame}

\end{document}