\documentclass[10pt,english]{beamer}
%\documentclass[english,handout]{beamer} % For handouts
\input{../metropolis_preamble.tex}
\input{../macros.tex}
%\usepackage{extendedalt}
%\usepackage{animate} % Animations
%\usepackage{../lindsten}
%\usepackage{movie15}
\usepackage{tikz}
\usepackage{listofitems} % for \readlist to create arrays

\title{732G12 Data Mining}
\subtitle{Föreläsning 7}
\date{}
\author{Johan Alenlöv \\ IDA, Linköping University, Sweden}
\titlegraphic{\hfill\includegraphics[height=1.2cm]{../LiU_primary_black.pdf}}
%\institute{Joint work with\dots}


%% MY DEF %%
\newcommand{\itm}[1]{\mathrm{Item}_{#1}}
\newcommand{\pausa}{\pause}
%\renewcommand{\pausa}{}
\tikzstyle{mynode}=[thick,draw=blue,fill=blue!20,circle,minimum size=22]


\newenvironment{nscenter}
 {\parskip=0pt\par\nopagebreak\centering}
 {\par\noindent\ignorespacesafterend}

\begin{document}

\maketitle

\begin{frame}{Dagens föreläsning}

    \begin{itemize}
        \item Introduktion
        \item K-means klustring
        \item Hierarkisk klustring
    \end{itemize}
    
\end{frame}

\begin{frame}{Introduktion}
    \begin{itemize}
        \item Oövervakad inlärning: Lära sig data \imp{utan} responsvariabel!
        \item Flera olika algoritmer:
        \begin{itemize}
            \item \imp{Klusteranalys}
            \item Associationsanalys
            \item Sekventiella mönster
            \item Dimensionalty reduction techniques
            \item PCS, Faktormodeller
            \item Representation Learning
        \end{itemize}
    \end{itemize}
\end{frame}

\begin{frame}{Introduktion}
    \begin{greenbox}
        Måler med klusteranalys är att dela upp datamaterialet i grupper (kluster) som är intressanta och/eller användbara.
    \end{greenbox}

    Vi vet inte i förväg vilka grupper som kommer att bildas.

    Ingen responsvariabel.
\end{frame}

\begin{frame}{Klusteranalys}

    Ta 5 minuter att fundera på följande frågor:

    \begin{enumerate}
        \item Hur många kluster finns det i bilden?
        \includegraphics[width = .7\textwidth]{figs/kluster1.png}
        \item Kom på något område där klusteranalys kan vara användbart.
    \end{enumerate}
    
\end{frame}

\begin{frame}{Klusteranalys}
    
    Ett "kluster" är inte entydigt definerat.

    \includegraphics[width = .7\textwidth]{figs/kluster1.png}

    Tillämpningsområden:
    \begin{itemize}
        \item Biologi (toxonomi/gener)
        \item Informationssökning (Sökmotorer)
        \item Psykologi och medicin
        \item Kunddata
        \item Sociala medier/nätverk
    \end{itemize}

\end{frame}

\end{document}