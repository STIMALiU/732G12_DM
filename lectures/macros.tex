\RequirePackage{amsmath, amssymb}
\RequirePackage{bbm}
%\RequirePackage{newtxmath}


% Convenience macro for referring to data source
\newcommand\sourceurl[2]{\small \grey{Data from \href{#1}{#2}}}

% Abbreviations
\RequirePackage{xspace}
\newcommand\pdf{pdf\xspace}
\newcommand\ifft{iff\xspace}
\newcommand\ex{\textbf{ex)}\xspace}

% General time series notation
\newcommand\T{n}  % Length of time series
\newcommand\rtheta{{\red{\theta}}}  % Parameter (color coded)
\newcommand\rthetah{{\red{\widehat\theta}}}  % Estimate (color coded)

% Neural netowkrs
\newcommand\h{\mathbf{h}} % Hidden state variable
\newcommand\zz{\mathbf{z}} % Generic input (vector)

% For OLS/AR
\newcommand\noise{\varepsilon}  % This is the noise in AR, but should it be the same as measurement noise in SSM?
\newcommand\noisevar{\sigma^2_\noise}
\newcommand\noisevarhat{\widehat\sigma^2_\noise}
\newcommand\X{\Phi}
\newcommand\y{\mathbf{y}}
\newcommand\bphi{\bm\phi}

% State space models
\newcommand\z{\alpha}  % State vector, general SSM
\newcommand{\obsnoise}{\varepsilon}
\newcommand{\statenoise}{\eta}
\newcommand{\varobs}{\sigma^2_{\varepsilon}}
\newcommand{\varstate}{\sigma^2_{\eta}}
% For structural time series
\newcommand{\trendnoise}{\zeta}
\newcommand{\seasnoise}{\omega}
\newcommand{\vartrend}{\sigma^2_{\trendnoise}}
\newcommand{\varseas}{\sigma^2_{\seasnoise}}

%
\newcommand\FF{T}
\newcommand\GG{R}
\newcommand\HH{Z}
\newcommand{\covobs}{\sigma_\epsilon^2}
\newcommand{\covstate}{Q}
\newcommand\initmean{a_1}
\newcommand\initcov{P_1}
% Kalman filter
\newcommand{\zpart}[2]{\z_{#1}^{#2}}
\newcommand{\wgt}[2]{\omega_{#1}^{#2}}
\newcommand{\wgtsum}[1]{\Omega_{#1}}
\newcommand\zhat[2]{\hat\z_{#1|#2}}
\newcommand\Phat[2]{P_{#1|#2}}
\newcommand\zpred[1]{\zhat{#1}{#1-1}}
\newcommand\Ppred[1]{\Phat{#1}{#1-1}}
\newcommand\zfilt[1]{\zhat{#1}{#1}}
\newcommand\Pfilt[1]{\Phat{#1}{#1}}
\newcommand\ypred[1]{\hat y_{#1|#1-1}}
\newcommand\Spred[1]{F_{#1|#1-1}}
\newcommand\Spredinv[1]{\Spred{#1}^{-1}}
\newcommand\epshat[2]{\hat{\obsnoise}_{#1|#2}}
\newcommand\etahat[2]{\hat{\statenoise}_{#1|#2}}

\newcommand{\statefun}{T}
\newcommand{\obsfun}{Z}
\newcommand{\estfun}{h}

\newcommand{\qd}{q} %State density
\newcommand{\md}{g} %Measure density

\newcommand{\rmd}{\mathrm{d}}

% SMC
\newcommand{\Np}{N}           % Number of particles
\newcommand{\Mp}{M}           % Number of particles in backward simulation



%\RequirePackage{color}
%\newcommand{\flnote}[1]{{\color{red}\textbf{[#1]}}} % Used for notes in text - color red
%\newcommand\Hrule{\vspace{1ex} \hrule \vspace{1ex}} % Horisontal rule with some space after; This is moved to beamer preamble

%%%%%%%%%%%%%%%%%%%%%%%%%%%%%%%%%%%%%%%%%%%%%%%%%%%%%%%%%%%%%%%%%%%%%%%%%%%%%%%%
%                            COMMANDS IN TEXT                                  %
%%%%%%%%%%%%%%%%%%%%%%%%%%%%%%%%%%%%%%%%%%%%%%%%%%%%%%%%%%%%%%%%%%%%%%%%%%%%%%%%
\newcommand\numtext[2]{#1\textsuperscript{#2}}
\newcommand\thsnd[1]{\ensuremath{#1\thinspace000}}
\newcommand{\peqref}[1]{\eqref{#1} on page~\pageref{#1}} % Page referencing for equations: "(1) on page 1"

%%%%%%%%%%%%%%%%%%%%%%%%%%%%%%%%%%%%%%%%%%%%%%%%%%%%%%%%%%%%%%%%%%%%%%%%%%%%%%%%
%                            SPECIFIC MATH                                     %
%%%%%%%%%%%%%%%%%%%%%%%%%%%%%%%%%%%%%%%%%%%%%%%%%%%%%%%%%%%%%%%%%%%%%%%%%%%%%%%%
% Models etc.
%\newcommand{\T}{T}            % Number of samples in data record
\newcommand{\parspace}{\Theta}                                   % Parameter space
\newcommand{\parameter}{\theta}                                  % Parameter
% Spaces
\newcommand{\setX}{\ensuremath{\mathsf{X}}}                      % State-space X
\newcommand{\sigmaX}{\ensuremath{\mathcal{X}}}                   % Sigma algebra on X
\newcommand{\setY}{\ensuremath{\mathsf{Y}}}                      % State-space Y
\newcommand{\sigmaY}{\ensuremath{\mathcal{Y}}}                   % Sigma algebra on Y
\newcommand{\setZ}{\ensuremath{\mathsf{Z}}}                      % State-space Z
\newcommand{\sigmaZ}{\ensuremath{\mathcal{Z}}}                   % Sigma algebra on Z

%%%%%%%%%%%%%%%%%%%%%%%%%%%%%%%%%%%%%%%%%%%%%%%%%%%%%%%%%%%%%%%%%%%%%%%%%%%%%%%%
%                           GENERAL MATH                                       %
%%%%%%%%%%%%%%%%%%%%%%%%%%%%%%%%%%%%%%%%%%%%%%%%%%%%%%%%%%%%%%%%%%%%%%%%%%%%%%%%

% ======== Miscellaneous symbols ========
\newcommand\eqdef{:=}
\newcommand\defeq{=:}
\newcommand\const{\text{const.}}
%\newcommand\eqdef{\stackrel{\text{\scriptsize def}}{=}}

\newcommand\iid{iid}
\newcommand{\iidsim}{\stackrel{\text{\iid}}{\sim}} % iid simulation
\newcommand{\process}[1]{\{#1\}_{t\geq 1}}       % Process (time index t)
\newcommand{\range}[2]{#1, \, \dots, \, #2}      % Range = 1, ..., N
\newcommand{\crange}[2]{\{#1, \, \dots, \, #2\}} % Curly range = {1, ..., N}
\newcommand{\prange}[2]{(#1, \, \dots, \, #2)}   % Parenthesised range = (1, ..., N)
\newcommand{\bwdrange}[2]{#1 : -1 : #2}          % Range = N, ..., 1
\newcommand{\approxpropto}{\stackrel{\sim}\propto}

% Tight dots between \int and \int in a multidimensional integral
\newcommand{\tightcdots}{\hspace*{-0.38em}\cdot\hspace*{-0.3em}\cdot\hspace*{-0.3em}\cdot\hspace*{-0.38em}}

% Arrows - convergence and mappings
% \mapsto                                                     % Mappings, x \mapsto f(x)
\newcommand{\fromto}{\rightarrow}                             % Mapping from set A to set B; f: A \fromto B
\newcommand{\goesto}{\rightarrow}                             % limits used in n \goesto \infty
\newcommand{\goestosmall}{\to}                                % limits used in \lim_{n \goestosmall \infty}
\newcommand{\convP}{\stackrel{\probab}\longrightarrow}        % Convergence in probability
\newcommand{\convD}{\stackrel{\textrm{D}}\longrightarrow}     % Convergence in distribution

% ======== Standard spaces  ========
\newcommand{\naturals}{\ensuremath{\mathbb{N}}}               % Natural numbers
\newcommand{\reals}{\ensuremath{\mathbb{R}}}                  % Real numbers
\newcommand{\nonnegatives}{\reals_{\smaller +}}               % Nonnegative numbers
\newcommand{\positives}{\reals_{\smaller ++}}                 % Positive numbers
\newcommand{\nonnegativedefinites}[1]{S_{\smaller +}(#1)}     % Nonnegative #1 x #1 matrices
\newcommand{\positivedefinites}[1]{S_{++}(#1)}                % Positive #1 x #1 matrices

% ======== Matrices ========
\newcommand{\eye}[1]{I_{#1}}                     % Identity matrix
\newcommand{\+}{\mathsf{T}}                      % Transpose
\newcommand{\kronecker}{\raisebox{1pt}{\ensuremath{\otimes}}} % Kronecker product
\DeclareMathOperator*\diag{diag}
\DeclareMathOperator*\trace{tr}

% ======== Operators, calculus etc. ========
\newcommand{\Ordo}{O}                            % Big ordo
\newcommand{\supnorm}[1]{\|#1\|_\infty}          % Supremum norm
\newcommand\osc{\text{osc}}                      % Oscillator norm
\newcommand{\grad}{\nabla}                       % Gradient
\newcommand{\complementof}[1]{\ensuremath{#1^\mathsf{c}}} % Set complement
\renewcommand\vec{\text{vec}}
\DeclareMathOperator*\supp{supp}                          % Support
\DeclareMathOperator*\card{card}                          % Set cardinality
\DeclareMathOperator*\rank{rank}                          % Rank
\DeclareMathOperator*\sign{sign}                          % Signum function
\DeclareMathOperator*\argmax{arg\,max}
\DeclareMathOperator*\argmin{arg\,min}

% ======== Probability ========
\newcommand{\Prb}{\ensuremath{\mathbb{P}}}                       % Probability
\newcommand{\E}{\ensuremath{\mathbb{E}}}                         % Expectation
\newcommand{\var}{\ensuremath{\mathrm{Var}}}                     % Variance
\newcommand{\cov}{\ensuremath{\mathrm{Cov}}}                     % Covariance
\newcommand{\cor}{\ensuremath{\mathrm{Corr}}}                     % Correlation
\newcommand{\I}{\ensuremath{\mathbbm{1}}}						 % Indicator function

%\newcommand{\abscont}{\ensuremath{\ll}}          % Absolute continuity
\renewcommand\mid{\,\vert\,} % I don't really like that \mid produces rubber lengths. Sometimes, we get very large white spaces p(x    |   y), and it can produce line breaks after "p(x |" . Is the non-rubber definition here better?
\newcommand\Mid{\,\middle\vert\,} % Stretchable |, to use with \left \right - N.B. This produces a longer | in general. Does that look better than a standard \mid?


% Distributions
\newcommand{\N}{\ensuremath{\mathcal{N}}}        % Normal
\newcommand{\uni}{\ensuremath{\mathcal{U}}}      % Uniform
\newcommand\MN{\mathcal{MN}}                     % Matrix normal
\newcommand\IW{\mathcal{IW}}                     % Inverse-Wishart
\newcommand\GP{\mathcal{GP}}                     % Gaussian process
\DeclareMathOperator*\Mult{Mult}                 % Multinomial
\DeclareMathOperator*\cat{Cat}                   % Categorical
\DeclareMathOperator*\Discrete{Discrete}         % Categorical/alternative name
\DeclareMathOperator*\bin{Bin}                   % Binomial
\DeclareMathOperator*\gam{Gam}                   % Gamma
\DeclareMathOperator*\St{St}                     % Student's t
\DeclareMathOperator*\po{Po}                   % Binomial
